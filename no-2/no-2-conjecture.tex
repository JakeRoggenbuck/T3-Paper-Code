\documentclass[12pt]{article}

\usepackage{listings}
\usepackage{color}

\newtheorem{theorem}{Theorem}[section]
\newtheorem{corollary}{Corollary}[theorem]
\newtheorem{lemma}[theorem]{Lemma}

\definecolor{mygreen}{rgb}{0,0.6,0}
\definecolor{mygray}{rgb}{0.5,0.5,0.5}
\definecolor{mymauve}{rgb}{0.58,0,0.82}

\lstset{language=python, tabsize=4, frame=single, basicstyle=\ttfamily}

\lstset{ %
  backgroundcolor=\color{white},   % choose the background color
  basicstyle=\footnotesize,        % size of fonts used for the code
  breaklines=true,                 % automatic line breaking only at whitespace
  captionpos=b,                    % sets the caption-position to bottom
  commentstyle=\color{mygreen},    % comment style
  escapeinside={\%*}{*)},          % if you want to add LaTeX within your code
  keywordstyle=\color{blue},       % keyword style
  stringstyle=\color{mymauve},     % string literal style
}

\title{ No-2 Conjecture }

\begin{document}
\maketitle

\section{Introduction}
The \textit{No-2 Conjecture} says that there is no consecutive sequence of ones with a size equal to two in the sequence $T_3$. In other words, no sequence of T translates to a sequence of $T_3$ that has exactly two consecutive ones.\\

Note that this pattern of exactly two consecutive ones does exist in T, but not $T_3$.

\begin{lemma}
The smallest and only sequence that contains exactly two consecutive ones is 0110.
\end{lemma}

\begin{lemma}
There exists no consecutive sequences of ones or zeros greater than two in T.

E.g. 0110 and 1001 exist in T, but 1110 and 0100 do not.
\end{lemma}

\begin{theorem}
Any sequences of T that leads to a sequence of $T_3$ that contains two consecutive ones must have ones in locations 3, and 6.

E.g. . . . 1 . . 1 . . .
\end{theorem}

\begin{theorem}
Any sequences of T that leads to a sequence of $T_3$ that contains two consecutive ones must also have zeros in locations 0, and 9.

E.g. 0 . . 1 . . 1 . . 0
\end{theorem}

\begin{corollary}
The only sequences that translate to a sequence of $T_3$ that are equal to 0110 that follow Lemma 1.2, Theorem 1.2, and Theorem 1.3 are the following. \\

Possible T sequences that translates to $T_3$ equal to 0110 = ['0011001010',
 '0011001100',
 '0011011010',
 '0101001010',
 '0101001100',
 '0101011010',
 '0101101010',
 '0101101100']

\end{corollary}

\begin{lemma} For any sequence of length at most $2^n$, if the Thue-Morse sequence doesn't contain it in the first $2^{n+3}$ digits, it contains it nowhere.
\end{lemma}

\begin{corollary}
	None of the sequences from the list of possible T sequences that translates to $T_3$ equal to 0110 are contained in first $2 ^ {9 + 3} = 4096$ digits of T, and therefore exist nowhere in T.
\end{corollary}


\section{Code Verification}

\subsection{Possible T sequences that translates to $T_3$ equal to 0110 not found in first 4096 of T}
The first 4096 of T do not contain any of the 8 possible T sequences that translates to $T_3$ equal to 0110.\\

\begin{lstlisting}[caption={Possible T sequences that translates to $T_3$ equal to 0110 not found in first 4096 of T}]
first_4096 = ""

for x in range(0, 4096):
    first_4096 = "1" if bin(x).count('1') % 2 else "0"

possible = [
    '0011001010',
    '0011001100',
    '0011011010',
    '0101001010',
    '0101001100',
    '0101011010',
    '0101101010',
    '0101101100',
]

for x in possible:
    if x in first_4096:
        print("Found")
\end{lstlisting}

\subsection{Find possible T sequences that translates to $T_3$ equal to 0110}
\begin{lstlisting}[caption={Find possible T sequences that translates to $T_3$ equal to 0110}]
import itertools
from pprint import pprint

perms = [''.join(x) for x in itertools.product('01', repeat=10)]

def remove_less_than_two_in_a_row_t3(items: list[str]):
    exactly_two = []
    for x in items:
        if x[3] == '1' and x[6] == '1':
            exactly_two.append(x)

    return exactly_two

def remove_more_than_two_in_a_row_t3(items: list[str]):
    exactly_two = []
    for x in items:
        t3 = ""
        for a in range(0, 10, 3):
            t3 += x[a]

        if "111" not in t3:
            exactly_two.append(x)

    return exactly_two

def remove_more_than_two_in_a_row_t_new(items: list[str]):
    left = []
    for item in items:
        if "111" not in item and "000" not in item:
            left.append(item)
    return left

perms = remove_more_than_two_in_a_row_t_new(perms)
perms = remove_less_than_two_in_a_row_t3(perms)
perms = remove_more_than_two_in_a_row_t3(perms)
pprint(perms)
\end{lstlisting}

\end{document}
