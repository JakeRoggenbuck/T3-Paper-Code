\documentclass{article}
\usepackage{amsmath, amssymb}
\begin{document}

\title{Exploring the Thue-Morse sequence}
\author{Adam Hutchings, Jake Roggenbuck}
\maketitle

\section{Introduction}
The Thue-Morse sequence can be defined in many ways, but it is the sequence whose $n$th term is the number of ones in the binary representation of $n.$ The sequence $T_3$ defined as the $n$th term being $T(3n)$ has some interesting properties that we will investigate.

\section{No-2 Theorem}
We define a $motif$ in $T_3$ to be a sequence of the same digit not contained inside any larger motif. The authors observed that for the first $10^11$ terms of $T_3,$ there are no motifs of length 2, which can be proven formally.

\textbf{No-2 Theorem} There are no motifs of length $2$ in $T_3.$

\textbf{Proof}
We begin by observing that a motif of zeros of length 2 is the sequence $00$ surrounded by $1$s, and a motif of ones of length 2 is the opposite. These translate to $1001$ and $0110$ in $T_3,$ which must exist if the No-2 Theorem is false. Because we obtain the elements of $T_3$ from choosing every third element of the Thue-Morse sequence, the existence of $1001$ or $0110$ in $T_3$ implies the existence of $1XX0XX0XX1$ or $0XX1XX1XX0$ in the Thue-Morse sequence where $X$ is either digit.

\textbf{Lemma 2.1} The sequences $000$ and $111$ do not exist in the Thue-Morse sequence.

\textbf{Proof} The non-existence of repetitions of three in the sequence is a well-known fact, but may be established as follows: jumping from $2n$ to $2n+1$ changes a $0$ to a $1$ and nothing else, which means that $T_{2n}$ and $T_{2n+1}$ are opposite. Therefore, the entire Thue-Morse sequence consists of repetitions of $01$ and $10,$ which allows for no $000$ or $111.$

Lemma 2.1 allows for a computer program to comb through all ten-digit sequences of zeros and ones, with the constraints that they follow the pattern $0XX1XX1XX0$ and do not contain $000$ or $111.$ This does not check any occurrences of $1001$ in $T_3,$ but this is not necessary.

\textbf{Lemma 2.2} A sequence $X$ exists in the Thue-Morse sequence \emph{iff} the sequence $X_i$ also exists, where $X_i$ is $X$ with $0$ and $1$ interchanged.

\textbf{Proof} Such a sequence $X$ will occur entirely inside the first $n$ digits of the sequence, where $n$ is an arbitrarily large power of 2. The \emph{next} $n$ digits are the same as the first $n$ with $0$ and $1$ interchanged, so the existence of $X$ in the first block is equivalent to the existence of $X_i$ in the second.

The only ten-digit sequences that satisfy these constraints are $0011001010,$ $0011001100,$ $0011011010,$$0101001010,$ $0101001100,$ $0101011010,$ $0101101010,$ and $0101101100.$

\textbf{Lemma 2.3} If $n$ is some power of 2 and the sequence $X$ is not more than $n$ digits long, then if $X$ does not occur in the first $8n$ digits of the Thue-Morse sequence, it never occurs.

\textbf{Proof} Call the first $n$ digits of the Thue-Morse sequence $A$, and $A_i$ (as defined above) $B.$ Then by the same logic as the proof of Lemma 2.2, the first $8n$ digits are $ABBABAAB.$ As the sequence $n$ is no more than $n$ digits long, any occurrence of it is either entirely within one $n$-long block or straddling two. Therefore, the only possible "environments" in which the pattern may occur are $A,$ $B,$ $AA,$ $AB,$ $BA,$ or $BB.$ All of these environments occur in $ABBABAAB,$ so if the pattern never exists there, it never exists anywhere.

Now, Lemma 2.3 means we only need to check the first 128 digits (as the 10-digit sequences are less than 16 long, so checking 128 is sufficient.) Checking using a computer program rules these out, so the No-2 Theorem is proven.

\end{document}
