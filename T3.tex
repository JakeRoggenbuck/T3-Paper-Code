\documentclass{article}
\usepackage{amsmath, amssymb}
\begin{document}

\title{Exploring the Thue-Morse sequence}
\author{Adam Hutchings, Jake Roggenbuck}
\maketitle

\begin{abstract}
We investigate some interesting properties of the sequence made up of every third term of the Thue-Morse sequence.
\end{abstract}

\tableofcontents

\section{Introduction}
The Thue-Morse sequence can be defined in many ways, but it is the sequence whose $n$th term ($T(n)$) depends on the number of ones in the binary representation of $n:$ 1 if odd, 0 if even. The sequence $T_3,$ which is defined as the sequence whose $n$th term is $T(3n),$ has some interesting properties that we will investigate -- among the most interesting of which are the persistent imbalance of zeros and ones and the lengths of runs in the sequence.

\section{Run-Length Of $T_3$}

\subsection{Motifs and the $\Lambda$ Function}
We define a $motif$ in $T_3$ to be a sequence of the same digit not contained inside any larger sequence made up of entirely the same digit. Similar to the computer science notion of \emph{run-length encoding}, we define a function $\Lambda$ acting on a sequence $x,$ whose $n$th member is called $\Lambda(x)(n),$ as follows:

- $\Lambda(x)(0)$ is the length of the motif in $x$ starting at $x_0$.

- $\Lambda(x)(n)$ is the length of the motif starting at $\sum_{k=0}^{n-1} \Lambda(x)(k).$

In other words, $\Lambda$ takes as input a sequence of numbers and transforms it into a sequence denoting the lengths of motifs in the input. We further define $\Lambda_n(x)$ as $\Lambda(\Lambda_{n-1}(x)),$ and abbreviate $\Lambda_1$ to $\Lambda$.

\subsection{$\Lambda(T_3)$}

The authors created a computer program to investigate the sequence $\Lambda(T_3)$ and noted that it consists, for all values checked, of only 1, 3, 6, 7, and 8. In other words, motifs in $T_3$ consist of only these values.

\subsection{No-2 Theorem}

\textbf{No-2 Theorem} There are no motifs of length $2$ in $T_3.$
\newline
\textbf{Alternative Statement of the No-2 Theorem} $\Lambda(T_3)$ contains no $2$s.

\textbf{Proof}
We begin by observing that a motif of zeros of length 2 is the sequence $00$ surrounded by $1$s, and a motif of ones of length 2 is the opposite. These translate to $1001$ and $0110$ in $T_3,$ which must exist if the No-2 Theorem is false. Because we obtain the elements of $T_3$ from choosing every third element of the Thue-Morse sequence, the existence of $1001$ or $0110$ in $T_3$ implies the existence of $1XX0XX0XX1$ or $0XX1XX1XX0$ in the Thue-Morse sequence where $X$ is either digit.

\textbf{Lemma 2.1} The sequences $000$ and $111$ do not exist in the Thue-Morse sequence.

\textbf{Proof} The non-existence of repetitions of three in the sequence is a well-known fact, but may be established as follows: jumping from $2n$ to $2n+1$ changes a $0$ to a $1$ and nothing else, which means that $T_{2n}$ and $T_{2n+1}$ are opposite. Therefore, the entire Thue-Morse sequence consists of repetitions of $01$ and $10,$ which allows for no $000$ or $111.$ $\Box$

Lemma 2.1 allowed for us to create a computer program to comb through all ten-digit sequences of zeros and ones, with the constraints that they follow the pattern $0XX1XX1XX0$ and do not contain $000$ or $111.$  The only ten-digit sequences that satisfy these constraints are $0011001010,$ $0011001100,$ $0011011010,$$0101001010,$ $0101001100,$ $0101011010,$ $0101101010,$ and $0101101100.$ We do not check for any occurrences of $1001$ in $T_3,$ (which would follow the pattern $1XX0XX0XX1$ in $T$) but this is not necessary, as shown by:

\textbf{Lemma 2.2} A sequence $X$ exists in the Thue-Morse sequence \emph{iff} the sequence $\overline{X}$ also exists, where $\overline{X}$ is $X$ with $0$ and $1$ interchanged.

\textbf{Proof} Such a sequence $X$ will occur entirely inside the first $n$ digits of the sequence, where $n$ is an arbitrarily large power of 2. The \emph{next} $n$ digits are the same as the first $n$ with $0$ and $1$ interchanged, so the existence of $X$ in the first block is equivalent to the existence of $\overline{X}$ in the second. $\Box$

(This means that we do not need to check for any occurrences in $1001$ because these exist \emph{iff} occurrences of $0110$ occur.)

\textbf{Lemma 2.3} If $n$ is some power of 2 and the sequence $X$ is not more than $n$ digits long, then if $X$ does not occur in the first $8n$ digits of the Thue-Morse sequence, it never occurs.

\textbf{Proof} Call the first $n$ digits of the Thue-Morse sequence $A$, and $\overline{A}$ (as defined above) $B.$ Then by the same logic as the proof of Lemma 2.2, the first $8n$ digits are $ABBABAAB.$ As the sequence $n$ is no more than $n$ digits long, any occurrence of it is either entirely within one $n$-long block or straddling two. Therefore, the only possible "environments" in which the pattern may occur are $A,$ $B,$ $AA,$ $AB,$ $BA,$ or $BB.$ All of these environments occur in $ABBABAAB,$ so if the pattern never exists there, it never exists anywhere. $\Box$

Now, Lemma 2.3 means we only need to check the first 128 digits (as the 10-digit sequences are less than 16 long, so checking 128 is sufficient.) Checking using a simple computer program has ruled these out. $\Box$

\subsection{No-9+ Conjecture}
The authors observed that for the first $10^{10}$ terms of T3, there are not motifs of lengths 9 or greater, which can be proven formally.

\textbf{No-9+ Theorem} There are no motifs of length $9$ or greater in $T_3$.

\subsection{6,8 Synchronization Conjecture}
The authors observed that the runlengths of 1s of $T_3$ or $\Lambda{_1(T_3)}$ have a pattern in the order of 6s and 8s. It appears that 6 always follow 8 and vice versa however, there is a caveat. In $\Lambda{_1(T_3)}$, 6 does in fact follow 8 and vice versa until the location 87384 (\textbf{T OR T3?}) where there is an anomaly of a 6 following another 6. We have found by observation that frequency of anomalies stays between $0.019\%$ and $0.028\%$ from $10^7$ to $10^9$ in the digits of $\Lambda{_1(T_3)}$. The frequency of anomaly increases from the location 87384 forever.

\subsection{Uniform 3s Conjecture}
The frequency of values in $\Lambda{_1(T_3)}$ is uniform.

\section{The Imbalance Of Zeros And Ones}

The $T_3$ sequence begins $00000001000...,$ and the imbalance between zeros and ones continues on. We denote the ratio of zeros to ones in every third digit of a stretch of the Thue-Morse sequence in the first $n$ entries by $r(n),$ and note the following values calculated by a computer:

Ratio for the first $2^{10}$ digits: 2.8

Ratio for the first $2^{11}$ digits: 2.1045454545454545

Ratio for the first $2^{12}$ digits: 2.1045454545454545

Ratio for the first $2^{13}$ digits: 1.7282717282717284

Ratio for the first 2$^{14}$ digits: 1.7282717282717284

[Twenty entries omitted]

Ratio for the first $2^{35}$ digits: 1.0228080103552828

The ratio does indeed appear to approach $1,$ but only very slowly -- even at over $10^{10}$ entries of $T_3$ computed, there are still ~2\% more zeros than ones. Therefore, we conjecture the following:

\textbf{Ratio Conjecture} $r(n)$ is never less than $1.$

\subsection{The Weak Ratio Theorem}

\textbf{Weak Ratio Theorem} $r(2^k) > 1$ for all integers $k > 4.$

As a side note, the restriction $k > 4$ exists because the first one occurs at position $8$ in $T3,$ so $r(2^k)$ for $n < 5$ is undefined.

\textbf{Proof} To start, as in the proof of Lemma 2.3, we may consider that for some natural number $k,$ the entire Thue-Morse sequence is composed of two blocks of length $2^k,$ which we may call $A_k$ and $B_k,$ such that $B_k$ is $A_k$ with every $0$ and $1$ swapped. Next, we will consider three pairs of functions: $\alpha_{0/1},$ $\beta_{0/1},$ and $\gamma_{0/1}.$ $\alpha_0(k)$ is the number of zeros in the block $A_k$ when we count every third element, \emph{starting from element 1}, and $\alpha_1(k)$ is the number of ones when we count in the same manner. The $\beta$ functions are the totals when taken starting on the second element of the block $A_k,$ and the $\gamma$ functions are counted starting on the third elements. Based on the definition of the $T_3$ sequence, we see that:

$$r(2^k) = \frac{\alpha_0(k)}{\alpha_1(k)}.$$

We also note that because $\alpha_0(k) > 0$ and $\alpha_1(k) > 0$ for $k > 4,$ the Weak Ratio Theorem can be phrased in terms of a difference.

\textbf{Alternative Weak Ratio Theorem Statement} $\alpha_0(k) > \alpha_1(k)$ for all $k > 4.$

We may continue by deriving expressions for the values of each of these functions by stating them in terms of equivalent values for smaller blocks. Assuming $k$ is even, $2^k$ is equal to 1 \emph{mod} 3, and if $k$ is odd then $2^k$ is equal to 2 \emph{mod} 3. Now, considering $\alpha_0(k+1)$ for even $k,$ we know that we count through the two blocks $AB,$ and because block $A$ has a remainder of 1 \emph{mod} 3, we start counting zeros in block $B$ at the third element, which is equivalent to counting ones starting at the third element in $A$ (because $B$ is equal to $A$ with zeros and ones swapped.) Therefore, we see that
$$\alpha_0(k+1) = \alpha_0(k) + \gamma_1(k).$$

Using the same reasoning, for even $k:$
$$\alpha_0(k+1) = \alpha_0(k) + \gamma_1(k), \alpha_1(k+1) = \alpha_1(k) + \gamma_0(k),$$
$$\beta_0(k+1) = \beta_0(k) + \alpha_1(k), \beta_1(k+1) = \beta_1(k) + \alpha_0(k),$$
$$\gamma_0(k+1) = \gamma_0(k) + \beta_1(k), \gamma_1(k+1) = \gamma_1(k) + \beta_0(k).$$
And for odd $k:$
$$\alpha_0(k+1) = \alpha_0(k) + \beta_1(k), \alpha_1(k+1) = \alpha_1(k) + \beta_0(k),$$
$$\beta_0(k+1) = \beta_0(k) + \gamma_1(k), \beta_1(k+1) = \beta_1(k) + \gamma_0(k),$$
$$\gamma_0(k+1) = \gamma_0(k) + \alpha_1(k), \gamma_1(k+1) = \gamma_1(k) + \alpha_0(k).$$

\textbf{Lemma 3.1.1} For even $k,$ $\alpha_0(k-3) + \beta_1(k-3) > \alpha_1(k-3) + \beta_0(k+3)$ \emph{iff} $r(2^k) > 1.$

\textbf{Proof} Noting that $k$ is even, we see that
$$\alpha_0(k) = \alpha_0(k-1) + \beta_1(k-1) = \alpha_0(k-2) + \gamma_1(k-2) + \beta_1(k-2) + \alpha_0(k-2).$$
Expanding one more level, we get:
$$\alpha_0(k-3) + \beta_1(k-3) + \gamma_1(k-3) + \alpha_0(k-3) + \alpha_0(k-3) + \beta_1(k-3) + \beta_1(k-3) + \gamma_0(k-3).$$
Rearranging, and remembering that $\gamma = \gamma_0 + \gamma_1:$
$$\alpha_0(k) = 3 \alpha_0(k-3) + 3 \beta_1(k-3) + \gamma(k-3).$$
The expression for $\alpha_1(k)$ is similar, but all subscript zeros and ones are flipped:
$$\alpha_1(k) = 3 \alpha_1(k-3) + 3 \beta_0(k-3) + \gamma(k-3).$$
Therefore, we have:
$$ \alpha_0(k) - \alpha_1(k) = 3 \alpha_0(k-3) - 3 \alpha_1(k-3) + 3 \beta_1(k-3) - 3 \beta_0(k-3).$$
Dividing this expression by $3$ will not change its sign, so if
$$\alpha_0(k-3) - \alpha_1(k-3) + \beta_1(k-3) - \beta_0(k-3) > 0,$$
Then $r(2^k) > 1.$ Rearranging terms in the expansion gives us Lemma 3.1.1.

\textbf{Lemma 3.1.2} For odd $k,$ $\alpha_0(k-3) + \gamma_1(k-3) > \alpha_1(k-3) + \gamma_0(k-3)$ \emph{iff} $r(2^k) > 1.$

\textbf{Proof} We proceed in the same way as last time, except switching the use of the odd and even expansions.

We can now restate Lemmas 3.1.1 and 3.1.2 slightly, remembering the Alternative Weak Ratio Theorem Statement and replacing $k$ with $k+3$ and $k-3$ with $k:$

\textbf{Lemma 3.1.1 Restated} For odd $k,$ $\alpha_0(k) + \beta_1(k) > \alpha_1(k) + \beta_0(k)$ \emph{iff} $\alpha_0(k+3) > \alpha_1(k+3).$

\textbf{Lemma 3.1.2 Restated} For even $k,$ $\alpha_0(k) + \gamma_1(k) > \alpha_1(k) + \gamma_0(k)$ \emph{iff} $\alpha_0(k+3) > \alpha_1(k+3).$

We now remember that for odd $k,$ $\alpha_0(k+1) = \alpha_0(k) + \beta_1(k),$ and $\alpha_0(k+1) = \alpha_0(k) + \gamma_1(k)$ for even $k.$ Doing the analogous expansions for $\alpha_1(k),$ we can restate the two lemmas yet again.

\textbf{Lemma 3.1.1 Restated} For odd $k,$ $\alpha_0(k+1) > \alpha_1(k+1)$ \emph{iff} $\alpha_0(k+3) > \alpha_1(k+3).$

\textbf{Lemma 3.1.2 Restated} For even $k,$ $\alpha_0(k+1) > \alpha_1(k+1)$ \emph{iff} $\alpha_0(k+3) > \alpha_1(k+3).$

We can combine these two by forgetting the irrelevant odd/even distinction and replacing $k+1$ with $k$ and $k+3$ with $k+2:$

\textbf{Lemma 3.1.3} For all $k,$ $\alpha_0(k) > \alpha_1(k)$ \emph{iff} $\alpha_0(k+2) > \alpha_1(k+2).$

This means that we can show $\alpha_0(k) > \alpha_1(k)$ for $k = 5, 6$ and be done. In fact, we can check by hand:

$$\alpha_0(5) = 10, \alpha_1(5) = 1, \alpha_0(6) = 20, \alpha_1(6) = 2.$$

This completes the proof.

\end{document}
